\documentclass[12pt]{article}

\title{Building Intuition about K-L Divergence}

\usepackage{hyperref}
\usepackage{amsmath}
\usepackage{amssymb}
\usepackage{amsthm}

\newcommand{\E}{\mathbb{E}}
\newcommand{\N}{\mathcal{N}}

\newtheorem{proposition}{Proposition}

\begin{document}

\maketitle

\section{Definition}

The Kullback-Leibler
divergence\footnote{\url{https://en.wikipedia.org/wiki/Kullback-Leibler_divergence}}
of $Q$ from $P$ is defined as
\[ D_{KL}(P\|Q) = \int P(x) \left[ \log P(x) - \log Q(x) \right] dx , \]
where $P$ and $Q$ are probability densities with respect to a common
measure.  (In the case of counting measure, this makes $P$ and $Q$ be
probability mass functions.)  This is also an expectation over $P$
of a function involving logs of densities:
\[ D_{KL}(P\|Q) = \E_P (x \mapsto \log P(x) - \log Q(x)). \]

\section{Interpretations}

For discrete distributions, KL divergence can be interpreted as the
expected extra bits needed to code elements of $P$ with an optimal
code for $Q$, over what would be needed for an optimal code for $P$.

Freer, Mansinghka, and
Roy \footnote{\url{danroy.org/papers/FreerManRoy-NIPSMC-2010.pdf} When
  are probabilistic programs probably computationally tractable?}
prove that the KL divergence also gives the performance of a rejection
sampler.  Specifically,

\begin{proposition}[Freer, Mansinghka, Roy, pg. 2]
Let $N$ be the number of attempts before a rejection sampler for $P$
with proposal distribution $Q$ succeeds.  $N$ is geometrically distributed
with mean $\exp(D_{KL}(P\|Q))$.
\end{proposition}
This corresponds with the general intuition that proposal
distributions should try to be broader than their targets rather than
narrower.

\section{Properties}

KL divergnce is
\begin{itemize}
\item Non-negative, and zero only if $P = Q$ almost everywhere
\item Invariant to parameterization
\item \emph{Not} symmetric
\item Does \emph{not} obey the triangle inequality
\end{itemize}

For non-negativity, consider the function whose expected value is the
divergence: $x \mapsto \log P(x) - \log Q(x)$.  This function will be
(large and) positive in areas that are (much) more likely under $P$
than under $Q$, and (large and) negative in areas that are (much) more
likely under $Q$ than under $P$.  But since the expectation is being
taken with respect to $P$, it should stand to reason that the behavior
in areas likely under $P$ dominates.

For parameterization invariance, we reproduce the derivation from Wikipedia.
\begin{proof}
If a transformation is made from variable $x$ to variable $y(x)$,
then, since $P(x) dx = P(y) dy$ and $Q(x) dx = Q(y) dy$ the KL divergence
may be rewritten
\begin{eqnarray*}
 D_{KL}(P\|Q) & = & \int P(x) \log \left( \frac{P(x)}{Q(x)} \right) dx \\
  & = & \int P(y) \log \left( \frac{P(y)dy/dx}{Q(y)dy/dx} \right) dy \\
  & = & \int P(y) \log \left( \frac{P(y)}{Q(y)} \right) dy
\end{eqnarray*}
\end{proof}

\section{Example: KL of 1-D Gaussians}
\label{sec:example}

As a refresher, the probability density function of the Gaussian
distribution with mean $\mu$ and standard deviation $\sigma$ is

\[ \N(\mu,\sigma)(x) = \frac{1}{\sigma \sqrt{2 \pi}} \exp \left(-\frac{(x-\mu)^2}{2\sigma^2} \right), \]
and its log is correspondingly
\[ \log \N(\mu,\sigma)(x) = -\frac{(x-\mu)^2}{2\sigma^2} - \log \sigma - \log \sqrt{2\pi}. \]

Let us examine the behavior of the KL divergence of one Gaussian
distribution $Q$ from another, $P$.  By parameterization invariance,
we can normalize $P$ to be the standard Gaussian, which we will denote
$\N$, dropping the mean and deviation arguments.  Then we have

\begin{eqnarray*}
 D_{KL}(\N\|\N(\mu,\sigma)) & = & \E_\N \left[ \log \N(x) - \log \N(\mu,\sigma)(x) \right] \\
 & = & \E_\N \left[ -\frac{x^2}{2} + \frac{(x-\mu)^2}{2\sigma^2} + \log \sigma \right] \\
 & = & -\frac{1}{2} \E_\N[x^2] + \frac{1}{2\sigma^2} \E_\N[x^2] - \frac{2\mu}{2\sigma^2}\E_\N(x)
       + \frac{\mu^2}{2\sigma^2} + \log \sigma \\
 & = & \frac{\mu^2 + 1}{2\sigma^2} + \log \sigma - \frac{1}{2}.
\end{eqnarray*}
where we used that $\E_\N(x) = 0$ and $\E_\N(x^2) = 1$ are
the mean and variance of the standard Gaussian distribution.
Using translation and scaling, we get for the general case
\[ D_{KL}(\N(\mu_P,\sigma_P)\|\N(\mu_Q,\sigma_Q)) = \frac{(\mu_Q-\mu_P)^2 + \sigma_P^2}{2\sigma_Q^2} + \log \sigma_Q - \log \sigma_P - \frac{1}{2}. \]

Some interpretation:
\begin{itemize}
\item If $\sigma_Q$ is large relative to the other quantities, the
  $\log \sigma_Q$ term will dominate, so the divergence of a broad
  Gaussian from a narrow one grows logarithmically.
\item For fixed standard deviations that are small relative to the
  mean difference, the KL divergence of two Gaussians is quadratic in
  said mean difference.
\item If $\sigma_P$ is large relative to the other quantities, the KL
  divergence is quadratic in it.
\end{itemize}

\section{Mixture distributions}

Consider the definition of KL divergence again
\[ D_{KL}(P\|Q) = \E_P (x \mapsto \log P(x) - \log Q(x)), \]
this time in the context of $P$ and $Q$ being mixture distributions
\[ P(x) = \sum_i w_iP_i(x) \qquad Q(x) = \sum_j w_jQ_j(x) \qquad \sum_iw_i = \sum_jw_j = 1. \]

If the mixture components $P_i$ and widely separated, one
term will dominate each sum.  Call its index $i(x)$.
Then
\[ D_{KL}(P\|Q) \approx \E_P \left[ \log P_{i(x)}(x) + \log w_{i(x)} - \log Q(x) \right]. \]
The approximateness comes from the smaller terms in each sum, which
can be viewed as bumping up the weight of the best component a bit if
there is another component that's amost as good.\footnote{For
  instance, if $P_1(x) = P_2(x) \gg P_i(x)$, then $\log P(x) \approx
  \log P_1(x) + \log w_1 + \log w_2$.}  Moreover, the expectation over
$P$ is the weighted sum of expectations over $P_i$, and we may assume
that for $x$ drawn from $P_i$, $P_i$ stands to dominate the density
function.  This brings us to
\begin{eqnarray*}
 D_{KL}(P\|Q) & \approx & \sum_i w_i \log w_i + \sum_i w_i \E_{P_i} \left[ \log P_i(x) - \log Q(x) \right] \\
 & = & \sum_i w_i D_{KL}(P_i\|Q) - H(w_i).
\end{eqnarray*}
This is the weighted sum of divergences of $Q$ from the mixture
components $P_i$, less the entropy of the mixture weights $w_i$.

If the mixture components $Q_j$ are also widely separated, the same
reasoning applies to the $\log Q(x)$ term.  If we further assume that
for samples from any given component $P_i$, the contribution of a
single component $Q_{j(i)}$ always dominates, we find
\begin{eqnarray*}
 D_{KL}(P\|Q) & \approx & \sum_i w_i \log w_i + \sum_i w_i \E_{P_i} \left[ \log P_i(x) - \log Q_{j(i)}(x) - \log w_{j(i)} \right] \\
 & = & \sum_i w_i D_{KL}(P_i\|Q_{j(i)}) - H(w_i) + H(w_i, w_{j(i)}),
\end{eqnarray*}
which is the weighted (by the weights in $P$) sum of the KL divergences
of corresponding mixture components, less the entropy of the weights
of $P$, plus the cross-entropy of the weights $w_{j(i)}$ with respect
to the $w_i$.\footnote{Well, not exactly, because the $w_{j(i)}$ needn't
sum to 1, but the notation serves as a mnemonic.}

\section{Computing KL divergence}

KL divergence of $Q$ from $P$ is an expectation over $P$ of a function
determined by the density functions of $P$ and $Q$.  Therefore, if $P$
is samplable and $P$ and $Q$ are assessable (resp. approximately
assessable), we can approximate the KL (resp. the KL of approximate
distributions) by Monte Carlo integration:

\begin{eqnarray*}
 x_i & \sim & P \qquad \textrm{$N$ samples from $P$} \\
 \widehat{D_{KL}} & = & \frac{1}{N} \sum_i \log P(x_i) - \log Q(x_i).
\end{eqnarray*}

The Central Limit Theorem implies that if the random variable obtained
by computing $\log P(x_i) - \log Q(x_i)$ for $x_i$ sampled from $P$
has finite mean $M$ and variance $\Sigma^2$, then as $N$ rises the
distribution of the above estimate will converge to the Gaussian
$\N(M, \Sigma / \sqrt N)$.  We can therefore estimate the error
of any particular computed mean as the sample variance of
$\{\log P(x_i) - \log Q(x_i)\}_1^N$.

Let's check this for the Gaussian case.  From
Section~\ref{sec:example}, the true mean $M$ is
\begin{eqnarray*}
 M & = & D_{KL}(\N\|\N(\mu,\sigma)) = \frac{\mu^2 + 1}{2\sigma^2} + \log \sigma - \frac{1}{2}.
\end{eqnarray*}
The variance of our estimator
is

\begin{eqnarray*}
 \Sigma^2 & = & \E_\N\left[ \log \N(x) - \log \N(\mu,\sigma)(x) - M \right]^2 \\
 & = & \E_\N \left[ -\frac{x^2}{2} + \frac{(x-\mu)^2}{2\sigma^2} + \log \sigma - M \right]^2.
\end{eqnarray*}
To perform this calculation, we are going to first expand both squares
and group the powers of $x$.  Letting
\begin{eqnarray*}
 A & = & -\frac{1}{2} + \frac{1}{2 \sigma^2} \\
 B & = & \frac{\mu}{\sigma^2} \\
 C & = & \frac{\mu^2}{2\sigma^2} + \log \sigma
\end{eqnarray*}
we have
\begin{eqnarray*}
 M & = & A + C \\
 \Sigma^2 & = & \E_\N \left[ A x^2 - B x + C - (A + C)\right]^2 \\
 & = & \E_\N \left[ A^2 x^4 - 2 A B x^3 + \left(-2A^2 + B^2\right) x^2 + 2 A B x + A^2 \right] \\
 & = & 3A^2 -2A^2 + B^2 + A^2 \\
 & = & 2A^2 + B^2 \\
 & = & \frac{1}{2} - \frac{1}{\sigma^2} + \frac{1 + 2\mu^2}{2\sigma^4},
\end{eqnarray*}
where line 4 relies on the known values for the moments of the
standard Gaussian distribution: $\E_\N(1) = 1$, $\E_\N(x) = 0$,
$\E_\N(x^2) = 1$, $\E_\N(x^3) = 0$, $\E_\N(x^4) = 3$.

I read the above formulas as good news and bad news.  The good news is
that the variance of our KL estimator is finite for all divergences of
non-degenerate Gaussians, so the CLT applies and the estimate will
eventually converge to having a predictable error.  The bad news is
that this may take a \emph{long} time: the variance is quadratic in
the difference of means (when that difference begins to exceed the
standard deviation of $P$), and \emph{quartic} in the ratio of
standard deviations.  Comparing this to the formula for the mean
estimate itself, getting down to 10\% relative error means taking
$O(100 / \sigma^2)$ samples in the small-$\sigma$ regime.
\end{document}
