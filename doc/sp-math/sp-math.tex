\documentclass[12pt]{article}

\usepackage{amsmath}
\usepackage{hyperref}
\usepackage{parskip}

\title{Mathematical Derivations For Stochastic Procedures}

\begin{document}
\maketitle

\begin{abstract}
This TeX file serves as a location to collect reference formulas and
derivations for Venture SP implementations.  Ideally, premises of
derivations would be copied in from cited sources verbatim, so we can
check the correspondence easily (and don't depend on any transient
sources to stick around).  Also ideally, a given piece of code should
be as direct a translation of the conclusion of a derivation as
possible---same parameterization, same variable naming, etc.  As of
this writing, this document is far from complete.
\end{abstract}

\section{Continuous}

\subsection{Non-conjugate normal with sufficient statistics}
\texttt{SuffNormalOutputPSP}

\subsubsection{Log density of counts}
\href{http://www.encyclopediaofmath.org/index.php/Sufficient_statistic}
{The Encyclopedia of Math}
says that the likelihood of a vector $\{x_i\}$ of $n$ Gaussian
observations with given mean $\mu$ and variance $\sigma^2$ is

\[ p_{\mu,\sigma^2}(\{x_i\}) = (2\pi\sigma^2)^{-n/2} \exp\left( -\frac{n \mu^2}{2 \sigma^2}
   - \frac{1}{2 \sigma^2} \sum_{i=1}^n x_i^2 + \frac{\mu}{\sigma^2}\sum_{i=1}^nx_i\right). \]
\newcommand{\xsum}{x_{\textrm{sum}}}
\newcommand{\xsumsq}{x_{\textrm{sumsq}}}
Therefore, letting
\[ \xsum = \sum_{i=1}^nx_i \qquad \xsumsq = \sum_{i=1}^n x_i^2 \]
we have
\[ \log p_{\mu,\sigma^2}(\{x_i\}) = -\frac{n}{2}(\log(2\pi) + 2\log(\sigma))
   -\frac{n \mu^2}{2 \sigma^2}
   - \frac{1}{2 \sigma^2} \xsumsq + \frac{\mu}{\sigma^2}\xsum. \]

\subsubsection{Gradient of log density of counts}

Gradient with respect to $\mu$, $\sigma$, attached at \texttt{MakerSuffNormalOutputPSP}.

\begin{eqnarray*}
\frac{d}{d\mu} \log p_{\mu,\sigma^2}(\{x_i\}) & = & -\frac{n\mu}{\sigma^2} +
    \frac{\xsum}{\sigma^2}.
\end{eqnarray*}
To differentiate with respect to $\sigma$ it helps to group terms.  Letting
\newcommand{\deviance}{x_{\textrm{sq-dev}}}
\[ \deviance = \sum_{i=1}^n (x_i - \mu)^2 = n\mu^2 + \xsumsq - 2 \mu \xsum, \]
we have
\[ \log p_{\mu,\sigma^2}(\{x_i\}) = -\frac{n}{2}(\log(2\pi) + 2\log(\sigma))
   - \frac{\deviance}{2\sigma^2} \]
and
\[ \frac{d}{d\sigma} \log p_{\mu,\sigma^2}(\{x_i\}) = -\frac{n}{\sigma} +
    \frac{\deviance}{\sigma^3} \]

This cross-checks with
\href{http://aleph0.clarku.edu/~djoyce/ma218/meeting12.pdf}{this reference} (now broken).

\subsubsection{Upper bound of log density of counts}

Setting the above to zero and solving for $\mu$, $\sigma$ gives (for $n > 0$)
\begin{eqnarray*}
 \hat\mu & = & \frac{\xsum}{n} \\
 \hat\sigma^2 & = & \frac{1}{n}\left(n\hat\mu^2 + \xsumsq -2\hat\mu\xsum\right) \\
  & = & \frac{\xsum^2}{n^2} + \frac{\xsumsq}{n} - 2\frac{\xsum^2}{n^2} \\
  & = & \frac{\xsumsq}{n} - \frac{\xsum^2}{n^2}
\end{eqnarray*}

\section{Chinese Restaurant Process}

\texttt{CRPOutputPSP}

The Chinese Restaurant Process (CRP) is a two-parameter family indexed by
real numbers $(\alpha, \theta)$ satisfying either

\[
0 \le \alpha < 1, \theta > - \alpha
\]

or

\[
\alpha < 0, \theta = L\alpha
\]

for some $L \in \{1,2,\dots\}$. The stochastic procedure and mathematical
derivations below will only consider the \textit{first} requirement and
ignore the second one, for ease of exposition and availability of references.

We will let $[N] = \{1,\dots,N\}$ be the set which we are partitioning (customers),
$K$ be the number of blocks in the partition (tables) written as an ordered sequence
$(B_1,\dots,B_K)$, and $N_k$ be the cardinality $|B_k|$, $k = 1,\dots,K$
(number of customers at table $k$). Since the CRP is exchangeable, the probability
$\mathbf{P}(B|\alpha,\theta)$ of any partition is a function only of the statistics
$(N,\{N_k\},K)$.

\subsubsection{Log density of counts}

According to Equation (6) in
\href{http://arxiv.org/pdf/0909.3642.pdf}{Gndedin, et al}
for any partition $B=(B_1,\dots,B_K)$ of $[N]$ we have

\textbf{General Case}

\begin{equation}
\mathbf{P}(B|\alpha,\theta)
:= \frac{\Pi_{i=1}^{K-1}(\theta+i\alpha)}{(\theta+1)_{N-1}}\Pi_{j=1}^K(1-\alpha)_{N_k-1}
\label{eq:crp-density}
\end{equation}

where $(x)_n := x(x+1)\dots(x+n-1) = \frac{\Gamma(x+n)}{\Gamma(x)}$.

We will show that this formula recovers the simple one-parameter CRP which is more
common in the literature where $\alpha = 0$ (unfortunately, most references use
$\alpha$ for what we denote as $\theta$, but we have chosen the original notation from
\href{http://link.springer.com/article/10.1007/BF01213386}{Pitman} so that our
derivations math can cross-check easily with references on the two-parameter family).

\textbf{Case 1} $\alpha = 0$

\begin{align*}
\mathbf{P}(B|\alpha=0,\theta)
    & = \frac{\Pi_{i=1}^{K-1}(\theta)}{(\theta+1)_{N-1}}\Pi_{k=1}^K(1)_{N_k-1}\\
   & = \frac{\theta^{K-1}}{\frac{\Gamma(\theta+N)}{\Gamma(\theta+1)}}\Pi_{k=1}^K\Gamma(N_k)\\
   & = \frac{\theta^{K-1}}{\frac{\Gamma(\theta+N)}{\theta\Gamma(\theta)}}\Pi_{k=1}^K\Gamma(N_k)\\
   & = \frac{\theta^{K}}{\frac{\Gamma(\theta+N)}{\Gamma(\theta)}}\Pi_{k=1}^K\Gamma(N_k)\\
\end{align*}

which cross checks with Equation (8) from
\href{http://web.mit.edu/sjgershm/www/GershmanBlei12.pdf#page=4}{Gershman, et al},
using the fact that $\Gamma(n) = (n-1)!$ for integer $n$.

\textbf{Case 2} $\theta = 0$

\begin{eqnarray*}
&\mathbf{P}(B|\alpha,\theta=0)
    & = \frac{\Pi_{i=1}^{K-1}(i\alpha)}{(1)_{N-1}}\Pi_{j=1}^K(1-\alpha)_{N_k-1}\\
&   & = \frac{\alpha^{K-1}(K-1)!}{\Gamma(N)}\Pi_{j=1}^K
            \frac{\Gamma(N_k-\alpha)}{\Gamma(1-\alpha)}\\
&   & = \frac{\alpha^{K-1}\Gamma(K)}{\Gamma(N)}\Pi_{j=1}^K
\frac{\Gamma(N_k-\alpha)}{\Gamma(1-\alpha)}
\end{eqnarray*}

The only available reference with which this result cross checks is the
\href{https://en.wikipedia.org/wiki/Chinese_restaurant_process#Generalization}
{Wikipedia entry} on the CRP.

\subsubsection{Gradient of log density of counts}

Gradient with respect to $\theta, \alpha$ attached at \texttt{MakeCRPOutputPSP}.

Since the general density in \eqref{eq:crp-density} is quite complex, we will derive
only the gradient for \textbf{Case 1} which is the traditional CRP,
and leave both \textbf{General Case} and \textbf{Case 2} for a
future exposition.

\textbf{General Case} TODO

\textbf{Case 1} $\alpha = 0$

From the derivation, it follows directly that

\begin{align*}
\ln\mathbf{P}(B|\alpha=0,\theta)
    & = K\ln\theta + \sum_{k=1}^{K}\ln\Gamma(N_k) -(\ln\Gamma(\theta+N)-\ln\Gamma(\theta))\\
\implies \partial_\theta\ln\mathbf{P}(B|\alpha=0,\theta)
    &= \frac{K}{\theta} - (\psi(\theta+N) - \psi(\theta))
\end{align*}

where $\psi(x) = \partial_x\ln\Gamma(x)$ is the digamma function, the logarithmic
derivative of the gamma function.

\textbf{Case 2} $\theta = 0$ TODO
\end{document}
